\section{Geometrie}
\subsection{Kurvendarstellungen \bronstein{50}}
\begin{tabular}{ l l }
	Parameterdarstellung: & $\vec{c}(t) = \begin{pmatrix}
		x(t) \\
		y(t) \\
		\vdots
	\end{pmatrix} \quad (t \in \text{Intervall})$\\
	Explizite: & $y = f(x)$\\
	Implizite Darstellung: & $0 = y - x$    
\end{tabular}

\subsection{Umrechnung}
$r, x, y$ in entsprechende Formel einsetzen und vereinfachen:\\

\noindent Kart. $\rightarrow$ Polar: \\
$r^2 = x^2 + y^2 \qquad x = r\cos\varphi \qquad y = r\sin\varphi$\\

\noindent Polar. $\rightarrow$ Kart: \\
$x = r\cos\varphi \qquad y = r\sin\varphi$\\


\noindent Polar. $\rightarrow$ Param: \\
$\begin{pmatrix}
	x(\varphi) \\ y(\varphi)
\end{pmatrix} 
=
\begin{pmatrix}
	r(\varphi)\cos(\varphi) \\ r(\varphi)\sin(\varphi)
\end{pmatrix}$\\

\subsection{Kurven zweiter Ordnung \bronstein{204ff}}
\textbf{\href{https://www.youtube.com/watch?v=ZOOu9HqDFx0}{Hinweis:}} Bei Ableitungen nach zB $\dot{x}$, sind $y(x)$ immer auch von $x$ abhängig und müssen daher mit der inneren Ableitung multipliziert werden. 
\begin{align*}
	3x^2 + 2y^2 &= x + 4y \qquad \left| \cdot \frac{d(...)}{dx}\right. \\
	6x + 4y\dot{y} &= 1 + 4\dot{y}
\end{align*}

\subsection{Formeln}
\begin{center}
	\rotatebox{90}{
	\bgroup
	\def\arraystretch{2.5}
	\begin{tabular}[h]{l|c|c|c}
		& \textbf{Kart.} & \textbf{Para.} & \textbf{Polar} \\
		\toprule
		Steigung & $f'(x)$ & $\frac{\dot y}{\dot x}$ & $\frac{r'\sin \varphi + r \cos\varphi}{r'\cos \varphi - r\sin\varphi} $ \\
		\hline
		Bogenlänge & $\int_{x_0}^{x_1}\sqrt{1 + f'(x)^2}dx$ & $\int_{t_0}^{t_1}\sqrt{\dot x^2 + \dot y^2}dt$ & $\int_{\varphi_0}^{\varphi_1}\sqrt{r'^2 + r^2}d\varphi$ \\
		\hline
		Krümmung & $\frac{f''(x)}{\sqrt{1 + f'(x)^2}^3}$ & $\frac{\ddot y \dot x - \dot y \ddot x}{\sqrt{\dot x^2 + \dot y^2}^3}$ & $\frac{2\dot r^2 - r \ddot r + r^2}{\sqrt{\dot r^2 + r^2}^3}$ \\
		\hline
		Fläche & $\int_{a}^{b}f(x)dx$ & $\frac{1}{2}\int_{t_0}^{t_1}[x\dot y - \dot xy]dt$ & $\frac{1}{2}\int_{\varphi_0}^{\varphi_1} r^2 d\varphi$\\
		\hline
		Volumen um $\vec{x}^*$ & $\pi \int_{a}^{b}f(x)^2dx$ & $\left|\pi \int_{t_0}^{t_1}y^2 \cdot \dot xdt\right|$ & $\left|\pi \int_{\varphi_0}^{\varphi_1}r^2\sin^2\varphi \cdot (\dot r \cos\varphi - r\sin\varphi)d\varphi\right|$ \\
		\hline
		Oberfläche um $\vec{x}^*$ & $2\pi \int_{a}^{b}\left|f(x)\right|\cdot\sqrt{1 + f'(x)^2}dx$ & $\left|2\pi\int_{t_0}^{t_1}\left|y\right|\cdot\sqrt{\dot x^2 + \dot y^2}dt\right|$ & $\left|2\pi \int_{\varphi_0}^{\varphi_1}\left|r\sin\varphi\right|\cdot\sqrt{r^2 + \cdot r^2}d\varphi\right|$
	\end{tabular}
	\egroup
}
\end{center}
\noindent $^*$) Für Rotation um $\vec{y}$ können in Para. die Funktionen $y$ durch $x$ ersetzt werden. Für Kart. kann die Umkehrfunktion $f^{-1}(x)$ verwendet werden.\\\textbf{Achtung}: Integral Grenzen müssen in Kart. auch angepasst sein:
\begin{align*}
	f(x) = \frac{1}{3}x^2 &\xRightarrow[]{x \leftrightarrow y} f^{-1}(y) = \sqrt{3y} \\
	\int_{a}^{b} (\cdots) dx  &\xRightarrow[]{} \int_{f^{-1}(a)}^{f^{-1}(b)} (\cdots) dy
\end{align*}
