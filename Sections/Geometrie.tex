\section{Geometrie}
\subsection{Kurvendarstellungen \bronstein{50}}
\begin{tabular}{ l l }
	Parameterdarstellung: & $\vec{c}(t) = \begin{pmatrix}
		x(t) \\
		y(t) \\
		\vdots
	\end{pmatrix} \quad (t \in \text{Intervall})$\\
	Explizite: & $y = f(x)$\\
	Implizite Darstellung: & $0 = y - x$    
\end{tabular}

\subsection{Umrechnung}
$r, x, y$ in entsprechende Formel einsetzen und vereinfachen:\\

\noindent Kart. $\rightarrow$ Polar: \\
$r^2 = x^2 + y^2 \qquad x = r\cos\varphi \qquad y = r\sin\varphi$\\

\noindent Polar. $\rightarrow$ Kart: \\
$x = r\cos\varphi \qquad y = r\sin\varphi$\\


\noindent Polar. $\rightarrow$ Param: \\
$\begin{pmatrix}
	x(\varphi) \\ y(\varphi)
\end{pmatrix} 
=
\begin{pmatrix}
	r(\varphi)\cos(\varphi) \\ r(\varphi)\sin(\varphi)
\end{pmatrix}$\\

\subsection{Kurven zweiter Ordnung \bronstein{204ff}}
\todo{Polar und Parameterdarstellung}

\subsection{Formeln}
\begin{center}
	\rotatebox{90}{
	\bgroup
	\def\arraystretch{2.5}
	\begin{tabular}[h]{l|c|c|c}
		& \textbf{Kart.} & \textbf{Para.} & \textbf{Polar} \\
		\toprule
		Steigung & $f'(x)$ & $\frac{\dot y}{\dot x}$ & $\frac{r'\sin \varphi + r \cos\varphi}{r'\cos \varphi - r\sin\varphi} $ \\
		\hline
		Bogenlänge & $\int_{x_0}^{x_1}\sqrt{1 + f'(x)^2}dx$ & $\int_{t_0}^{t_1}\sqrt{\dot x + \dot y}dt$ & $\int_{\varphi_0}^{\varphi_1}\sqrt{r'^2 + r^2}d\varphi$ \\
		\hline
		Krümmung & $\frac{f''(x)}{\sqrt{1 + f'(x)^2}^3}$ & $\frac{\ddot y \dot x - \dot y \ddot x}{\sqrt{\dot x + \dot y}^3}$ & $\frac{2\dot r^2 - r \ddot r + r^2}{\sqrt{\dot r^2 + r^2}^3}$ \\
		\hline
		Fläche & $\int_{a}^{b}f(x)dx$ & $\frac{1}{2}\int_{t_0}^{t_1}[x(t)\dot y(t) - \dot x(t)y(t)]dt$ & $\frac{1}{2}\int_{\varphi_0}^{\varphi_1} r^2 d\varphi$\\
		\hline
		Volumen & $\pi \int_{a}^{b}f(x)^2dx$ & $\left|\pi \int_{t_0}^{t_1}y(t)^2\dot x(t)dt\right|$ & $\left|\pi \int_{\varphi_0}^{\varphi_1}r^2\sin^2\varphi \cdot (\dot r \cos\varphi - r\sin\varphi)d\varphi\right|$ \\
		\hline
		Oberfläche & $2\pi \int_{a}^{b}\left|f(x)\right|\cdot\sqrt{1 + f'(x)^2}dx$ & $\left|2\pi\int_{t_0}^{t_1}\left|y(t)\right|\cdot\sqrt{\dot x^2 + \dot y^2}dt\right|$ & $\left|2\pi \int_{\varphi_0}^{\varphi_1}\left|r\sin\varphi\right|\cdot\sqrt{r^2 + \cdot r^2}d\varphi\right|$
	\end{tabular}
	\egroup
}
\end{center}

\subsubsection{Drehachse}
Für Volumen Berechnungen ist es Zentral die Drehachse zu bestimmen. Mithilfe der Umkehrfunktion kann diese von $x$ nach $y$ gewechselt werden. ACHTUNG: Integralgrenzen müssen auch angepasst werden.
\todo{Beispiel für Grenzen}
