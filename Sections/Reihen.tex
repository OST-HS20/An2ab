\section{Reihen}
\subsection{Reihen \bronstein{1077}}
\noindent Harmonische Reihe
\[
HR_n = \sum \frac{1}{n} = 1 + \frac{1}{2} + \frac{1}{3} + \dots + \frac{1}{n} \approx \ln(n) + \gamma + \frac{1}{2n} \xRightarrow[]{n\rightarrow\infty}\infty
\]

\noindent Geometrische Reihe
\[
GR_n = \sum q^n = q^0 + q^1 + q^2 + \cdots + q^n = \frac{q^{n+1}-1}{q-1} \xRightarrow[]{n\rightarrow\infty}\frac{1}{1-q}
\]

\noindent Potenz Reihe\label{potenzreihe}
\[
PR_n = \sum\limits_{n=0}^{\infty} a_n(x -x_0)^n \qquad \begin{array}{ll}
	 x_0: \text{ Entwicklungsstelle} \\
	 a_n: \text{ Koeffizienten}
\end{array}
\]

\subsection{Konvergenz/Divergenz}
\subsubsection{Raffsumme}
Summe von benachbarten Differenzen $a_{n+1} - a_n$. Nur wenn letzter Summand $a_{n+1} \rightarrow 0$, dann konvergiert die Reihe.
Falls $a = \lim\limits_{n \rightarrow \infty} \neq 0$, dann Divergenz. $a=0$ funktioniert nicht!

\subsubsection{Leibniz-Reihe}
Konvergenz zeichnet sich durch folgende Eigenschaften aus:
\begin{enumerate}[nosep]
	\item Vorzeichenwechsel
	\item Letzter Summand $\rightarrow 0$
	\item $\left|a_n\right|$ monoton fallend
\end{enumerate}

\subsubsection{Cauchy-Wurzelkriterium}\label{W-Kriterium}
\[\alpha = \lim\limits_{n \rightarrow \infty}{\sqrt[n]{\left|a_n\right|}}\qquad
\left\lbrace
\begin{array}{r@{}l}
	\alpha < 1 &: \text{dann } \sum a_n \text{ Konvergenz} \\
	\alpha > 1 &: \text{dann } \sum a_n \text{ Divergenz} \\
	\alpha = 1 &: \text{Keine Aussage} \\
\end{array}\right. \]
Falls keine Aussage getroffen werden kann, siehe auch Leibniz, Maj./Min. oder Quotientenkrit.
\\
\noindent\textbf{Beispiel}:
\begin{align*}
	a_n &= \frac{1}{3} + \frac{1}{5^2} + \frac{1}{3^3} + \frac{1}{5^4} + \cdots \\
	\alpha &= \sqrt[n]{\left|a_n\right|} \xrightarrow{\text{Folge}} \frac{1}{3} + \frac{1}{5} + \frac{1}{3} + \frac{1}{5} + \cdots = \begin{cases}
		\tfrac{1}{3} &: n \text{ ungerade} \\
		\tfrac{1}{5} &: n \text{ gerade}
	\end{cases}  
\end{align*}
Kein $\lim\limits_{n\rightarrow\infty}$, aber \textbf{grösste} Teilfolge von $\tilde{\alpha} = \limsup\limits_{n\rightarrow\infty}\sqrt[n]{\left|a_n\right|}$ ergibt $\tilde{\alpha} = \tfrac{1}{3} \rightarrow \text{ Konvergenz}$

\subsubsection{D'Alembert Quotientenkriterium}
\[\alpha = \lim\limits_{n \rightarrow \infty}{\frac{a_{n+1}}{a_n}}\]
Siehe \verweiseref{W-Kriterium}, für $\alpha$ Auswertung.

\subsubsection{Integralkriterium}
Wenn das Integral konvergiert, dann auch die Reihe:
\[\int_{1}^{\infty}f(x)dx \quad\leq\quad \sum_{n=1}^{\infty}f(n)\]
Gilt nur wenn $f(x) \geq 0$ und $f$ fallend!

\subsubsection{Bedingt/Unbedingte Konvergenz}
\textbf{Bedingte Konvergenz}, durch Umordnen der Summanden können neue Summen enstehen, sogar Divergenz. Damit kann jede Zahl in $\mathbb{R}$ erreicht werden.\\
\textbf{Unbedingte Konvergenz} ändert die Summe nicht bei einer Umordung, kann geprüft werden mit der \textbf{Absoluten Konvergenz}:
$\sum\left|a_n\right| < \infty$

\subsubsection{Reihenentwicklung}
Um eine Entwicklung einer Reihe von einer Funktion zu finden, am einfachsten eine ähnliche existierende Reihe suchen und anpassen: $f(x) \xrightarrow[\text{entwicklung}]{\text{Reihen-}} \sum a_n(x - x_0)^n$.

\noindent Beispiel:
\begin{align*}
	f(x) &= \frac{2}{3+8x} &\xrightarrow{GM} \sum_{n=0}^{\infty}\frac{1}{1-q} \\
	&= 3\cdot \frac{1}{3(1 + \tfrac{8}{3}x)} \eqi \frac{2}{3}\cdot \frac{1}{1 - \underbrace{(-\tfrac{8}{3}x)}_q} \\
	&= \frac{2}{3}\sum_{n=0}^{\infty}\left(-\frac{8}{3}x\right)^n = \frac{2}{3}\sum_{n=0}^{\infty}(-1)^n \cdot \frac{8^n}{3^n}x^n
\end{align*}

\subsection{Fehlerabschätzung}
Worst-Case Abschätzung für \textbf{Leibniz-Reihe} durch Auswertung der Summe $s$ und Reihe $a$. Für andere Reihen zB Maj/Min verwenden:
\[\left|s - s_n\right| = \left|\sum_{k=n+1}^{\infty}s_n\right| \leq \underbrace{\left|a_{n+1}\right|}_\text{Worst-Case Fehler}\]

\noindent \textbf{Beispiel} $\sum_{k=1}^{\infty}(-1)^k\frac{1}{k!} = -0.6321$, Abschätzung nach $n=4$ Durchgängen: Fehler nach 4 Durchgängen ist noch höchstens $\tfrac{1}{120}$.

\subsection{Konvergenzradius}
Bestimmt den Bereich der Entwicklungsstelle $x_0$ welcher Konvergiert.
\noindent Für Konvergenz muss der Abstand $\left|x - x_0\right| < r$ sein, wenn $\left|x - x_0\right| > r$ dann Divergenz. Falls gleich, ist keine Aussage möglich. 
\[
\underbrace{\left|x - x_0\right|}_{\text{Abstand}} < \underbrace{\frac{1}{\lim\limits_{n \rightarrow \infty}\sqrt[n]{\left|a_n\right|}} \text{ -oder- } \frac{1}{\lim\limits_{n \rightarrow \infty}{\left|\frac{a_{n+1}}{a_n}\right|}}}_{\text{Radius } r}
\]

\noindent\textbf{Hinweis:} Quotientenkriterium kann nur verwendet werden, wenn $a_n$ konvergieren.


\noindent Siehe auch $PR_n$ Kapitel \ref{potenzreihe} oder \textbf{Taylor-Koeffizient}; 
\\$a_n$ ist der Koeffizient von der gegebenen Summe. Bsp: $\sum \underbrace{\frac{1}{m!^2}}_{a_n}(x - (\underbrace{- 1}_{ x_0}))^n \Rightarrow a_n\eqi \frac{f^{(n)}(x_0)}{n!}$ für die $x^n$-te Zahl.

\subsubsection{Konvergenzgeschwindigkeit}
Je kleiner der Radius $\left|x - x_0\right|$ ist, desto schneller ist die Konvergenz. \\
\noindent\textbf{Beispiel}:
\[
\ln(1+x) = \sum_{k=1}^{n}\frac{(-1)^{k-1}}{k} x^k + R_n
\]
\noindent Konkret $x=1$, $x_0 = 0$ und $n=5$ bedeutet: $\to \left|1 - 0\right| = \underbrace{1}_{\text{Radius}}$. Wenn nun $x = -\frac{1}{2}$, dann $\ln(1 + 1) = \ln(2) \eqi -\ln(\frac{1}{2})$ und Konvergenz ist viel schneller!


\subsubsection{Ableitung/Integrieren}
Beim Ableiten/Integrieren bleibt der Konvergenzradius erhalten! 
Die $i$-te Ableitung von $f(x)  = \sum_{n=0}^{\infty}a_nx^n$: 
\[	f^{(i)}(x) = \sum_{n=i}^{\infty}n(n-1)\cdot_{\dots}\cdot(n-i+1)a_nx^{n-i}\]
\noindent\textbf{Achtung}: Startindex muss angepasst werden! \\

Integration: \[\sum_{n=0}^{\infty}a_n\frac{x^{n+1}}{n+1}=\int f(x)dx+C\]
Hinweis: Für $C$, Konv.Radius-Zentrum $x_0$ einsetzen.