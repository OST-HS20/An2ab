\section{Differential Gleichungen}
Kurz \textbf{DGL}, beschreibt die Modellvar. $x$ und die Zielfunktionen $y$.

\subsection{Eigenschaften}
\[
b\cdot y''' + a(x) \cdot y'' + 0 \cdot y' = g(x)
\]
Die \textbf{Ordnung} ist die max. Ableitungsordnung $\to 3$ \\
Eine \textbf{Linear}-DGL beinhaltet alle $y;y';y'',...$ (Koeffizienten $\mathbb{R}$ können auch abhängig von $x$ sein $\to b \text{ und } a(x)$) mit $y^{(i)}$ \underline{Exponent $=1$}. $\to $ Linear \\
\textbf{Homogen} sind jene DGLs, welche Störfunktion $g(f) = 0$ haben. Andernfalls \textbf{Inhomogen}.


\subsection{DGL Analyse}
Mit \textbf{ABC-Analyse} überprüft, ob alle Lösungen von DGL gefunden wurden.\underline{ r.h.s} ist aufgelöst nach höchster Ordnung.\\

\noindent\textbf{Beispiel}:
$\underbrace{y'' = \overbrace{y}^\text{r.h.s.}}_\text{DGL} \qquad	\underbrace{y = a\sinh x + b\cosh x}_\text{Lösung}$\\
\textbf{A} Verifikation: Lösung in DGL einsetzen und überprüfen.
\begin{align*}
	y' &= a\cosh x + b\sinh x \\
	y'' &= a\sinh x + b\cosh x \textcolor{green}{ \eqi y}
\end{align*}
\textbf{B} Alle Anfangsbedingungen lösbar. Anfangswerte einsetzten und Determinate darf nicht $0$ sein: 
\begin{align*}
	\begin{pmatrix}
	y_0 = a\cosh x_0 + b\sinh x_0 \\
	y_1 = a\cosh x_0 + b\sinh x_0
	\end{pmatrix}
	\rightarrow
	\det\begin{pmatrix}
		\sinh x_0 & \cosh x_0 \\
		\cosh x_0 & \sinh x_0
	\end{pmatrix} = \textcolor{green}{ -1}
\end{align*}
\textbf{C} Eindeutigkeit (PL - Siehe Kapitel \ref{picarlindelöf}): \label{picarlindelöf_beispiel}
\begin{align*}
	1. &\qquad f = y \to \text{Stetig} \\
	2. &\qquad f_y = \frac{\delta f}{\delta y} = 1 ;\quad  f'_y = \frac{\delta^2 f}{\delta^2 y} = 0 \to \text{Keine Pollstellen}\\
	&\textcolor{green}{ \text{1. und 2. korrekt} \to PL}
\end{align*}

\subsection{Lösbarkeit - Picar-Lindelöf (PL)}\label{picarlindelöf}
DGL sind lösbar, wenn erstens r.h.s $f(x;y;...;y^{n-1})$ stetig lokal sind, dafür werden Anfangswerte in r.h.s eingesetzt. Zudem müssen zweitens alle Partiellen Ableitungen $\frac{\delta f}{\delta y} = f_{y^{(n-1)}}$ beschränkt (keine Pollstellen) sein, auch hier Anfangswerte einsetzen. Wenn eine der beiden Forderungen verletzt ist, dann ist \underline{keine Schlussfolgerung} möglich. Siehe \ref{picarlindelöf_beispiel} als Beispiel.

\subsubsection{Singularitäts-Test}
Wenn PL fehlschlägt, dann kann mit einem Singularitäts-Test eine mögliche Lösung gefunden werden. Dafür wird eine Funktion gesucht, welche r.h.s singulär werden lässt und diese wird in DGL eingesetzt.
\noindent\textbf{Beispiel:}
\begin{align*}
	y' &= -\frac{x}{2} + \sqrt{\frac{x^2}{4} + y} &\qquad y(2) = -1 \\
	f_y &= \frac{\delta f}{\delta y} = 0 + \frac{1}{2\sqrt{\frac{x^2}{4} + y}} \cdot 1  &\to \text{Pollstelle} \\
	\frac{x^2}{4} + y &= 0 \Rightarrow y = -\frac{x^2}{4} &\Rightarrow y' = -\frac{2x}{4} \textcolor{green}{\eqq} -\frac{x}{2} + \sqrt{0}
\end{align*}


\subsection{DGL-Lösungsverfahren}
\subsubsection{Seperation}
Ein Sonderfall für DGL 1. Ordnung ist die \underline{Seperation}. Wenn $x$ und $y$ seperiert auf einer Seite stehen können: $y' = g(x)\cdot y $ ($x$ durch Multiplikation von $y$ getrennt), dann kann mittels $\frac{y'}{y}$ und ein $\int_{x_0}^{x_1}dx$ die Funktion $y$ gefunden werden.

\subsubsection{Linearterm}
$y'$ = Formel $f$ in $z = ax + by + c$.
	\[z' = a + bf(z)\]


\subsubsection{Gleichgradigkeit}
$y'$ = Formel $f$ nur abhängig von $z = \frac{y}{x}$. 
\[
z' = \frac{1}{x}(f(z) - z) \quad;z_0 = \frac{y_0}{x_0}\]


\subsubsection{Linear DGL}
Um lineare DGL zu berechnen kann das charakteristische Polynom verwendet werden:
\[
\cdots + y'' + a_1y' + a_0y = f(x) \quad\rightarrow\quad \cdots + \lambda^2 + a_1\lambda + a_0 = 0
\]
aus $\lambda_{1,2}$ können die $\mathbb{H}$-Lösungen berechnet werden. Mittels Determinate: 
\begin{enumerate}[nosep]
	\item $D>0$: Starke Dämpfung\\$\mathbb{H} \eqi \left\{Ae^{\lambda_1x}+Be^{\lambda_2x} + \cdots \right\}_{A,B \in \mathbb{R}}$
	\item $D<0$: Schwache Dämpfung. Dabei gilt $\alpha = \Im(\lambda_{1,2 }) $ und \\ $\mathbb{H} \eqi \left\{Ae^{-\frac{a_1}{2}x}\cos(\alpha x)+Be^{-\frac{a_1}{2}x}\sin(\alpha x) + \cdots \right\}_{A,B \in \mathbb{R}}$
	\item $D=0$ aperiodischer Grenzfall\\$\mathbb{H} \eqi \left\{Ae^{\lambda_1x}+Be^{\lambda_1x}  + \cdots \right\}_{A,B \in \mathbb{R}}$
\end{enumerate}

\noindent Anschliessend können die $\mathbb{P}$-Lösungen mittels \textbf{Faltungsintegral} berechnet werden:
\[
y_p(x) = \int_{x_0}^{x}g(x + x_0 - t)\cdot\underbrace{f(t)}_\text{Störterm}dt
\]
\noindent Die Funktion $g\in\mathbb{H}$ mit Standartvektor "binär" $1 \to (0;0;\cdots;1)$. Konkret wird die $\mathbb{H}$-Lösung verwendet und mittels Anfangswerte $A,B,...$ berechnet und somit eine konkrete Funktion $g$ gefunden.\\

\noindent Als Alternative für die $\mathbb{P}$-Lösung kann 

\noindent\textbf{Beispiel:} Ordnung=$2: \qquad \Rightarrow g(0) \eqi 0 \quad g'(0) \eqi 1$\\

\noindent Die DGL Lösung ist immer eine Summe von $\mathbb{L} = \mathbb{H} + \mathbb{P}$
\subsection{Orthogonaltrajektorien}
Senkrechte Kurven, welche Originalkurven immer senkrecht schneiden. \\ 
(1) Ableiten und Parameter eliminieren, DGL als r.h.s notieren (2) Orthogonale Kurven mit Negative Reziprok definieren.\\
\noindent\textbf{Beispiel}:
\begin{align*}
	\text{(1) } y &= cx & | \cdot \frac{d}{dx}\quad | c=\cdots\\
	y' &= c \quad c = \frac{y}{x} \Rightarrow y' = \frac{y}{x}\\
	\text{(2) } \tilde{y}' &= -\frac{x}{y}
\end{align*}

