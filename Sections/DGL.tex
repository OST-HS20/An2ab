\section{Differential Gleichungen}
\subsection{Eigenschaften}
\[
b\cdot y''' + a(x) \cdot y'' + 0 \cdot y' = g(x)
\]
Die \textbf{Ordnung} ist die max. Ableitungsordnung $\to 3$ \\
Eine \textbf{Linear}-DGL beinhaltet alle $y;y';y'',...$ (Koeffizienten $\mathbb{R}$ können auch abhängig von $x$ sein $\to b \text{ und } a(x)$) mit $y^{(i)}$ \underline{Exponent $=1$}. $\to $ Linear \\
\textbf{Homogen} sind jene DGLs, welche Störfunktion $g(f) = 0$ haben. Andernfalls \textbf{Inhomogen}.


\subsection{DGL Analyse}
Mit \textbf{ABC-Analyse} überprüft, ob alle Lösungen von DGL gefunden wurden.\underline{ r.h.s} ist aufgelöst nach höchster Ordnung.\\

\noindent\textbf{Beispiel}:
$\underbrace{y'' = \overbrace{y}^\text{r.h.s.}}_\text{DGL} \qquad	\underbrace{y = a\sinh x + b\cosh x}_\text{Lösung}$\\
\textbf{A} Verifikation: Lösung in DGL einsetzen und überprüfen.
\begin{align*}
	y' &= a\cosh x + b\sinh x \\
	y'' &= a\sinh x + b\cosh x \textcolor{green}{ \eqi y}
\end{align*}
\textbf{B} Alle Anfangsbedingungen lösbar. Anfangswerte einsetzten und Determinate darf nicht $0$ sein: 
\begin{align*}
	\begin{pmatrix}
	y_0 = a\cosh x_0 + b\sinh x_0 \\
	y_1 = a\cosh x_0 + b\sinh x_0
	\end{pmatrix}
	\rightarrow
	\det\begin{pmatrix}
		\sinh x_0 & \cosh x_0 \\
		\cosh x_0 & \sinh x_0
	\end{pmatrix} = \textcolor{green}{ -1}
\end{align*}
\textbf{C} Eindeutigkeit (PL - Siehe Kapitel \ref{picarlindelöf}): \label{picarlindelöf_beispiel}
\begin{align*}
	1. &\qquad f = y \to \text{Stetig in } \mathbb{D} \\
	2. &\qquad f_y = \frac{\delta f}{\delta y} = 1 ;\quad  f'_y = \frac{\delta^2 f}{\delta^2 y} = 0 \to \text{Keine Pollstellen}\\
	&\textcolor{green}{ \text{1. und 2. korrekt} \to PL}
\end{align*}

\subsection{Lösbarkeit - Picar-Lindelöf (PL)}\label{picarlindelöf}
DGL sind \underline{eindeutig lösbar}, wenn erstens r.h.s $f(x;y;...;y^{n-1})$ stetig lokal sind, dafür werden Anfangswerte in r.h.s eingesetzt. Zudem müssen zweitens alle Partiellen Ableitungen $\frac{\delta f}{\delta y} = f_{y^{(n-1)}}$ beschränkt (keine Pollstellen) sein, auch hier Anfangswerte einsetzen. Wenn eine der beiden Forderungen verletzt ist, dann ist \underline{keine Schlussfolgerung} möglich. Siehe \ref{picarlindelöf_beispiel} als Beispiel.

\subsubsection{Singularitäts-Test}
Wenn PL fehlschlägt, dann kann mit einem Singularitäts-Test eine mögliche Lösung gefunden werden. Dafür wird eine Funktion gesucht, welche r.h.s singulär werden lässt und diese wird in DGL eingesetzt.
\noindent\textbf{Beispiel:}
\begin{align*}
	y' &= -\frac{x}{2} + \sqrt{\frac{x^2}{4} + y} &\qquad y(2) = -1 \\
	f_y &= \frac{\delta f}{\delta y} = 0 + \frac{1}{2\sqrt{\frac{x^2}{4} + y}} \cdot 1  &\to \text{Pollstelle} \\
	\frac{x^2}{4} + y &= 0 \Rightarrow y = -\frac{x^2}{4} &\Rightarrow y' = -\frac{2x}{4} \textcolor{green}{\eqq} -\frac{x}{2} + \sqrt{0}
\end{align*}


\subsection{DGL-Lösungsverfahren}
\subsubsection{Seperation}
Ein Sonderfall für DGL 1. Ordnung ist die \underline{Seperation}. Wenn $x$ und $y$ seperiert auf einer Seite stehen können: $y' = g(x)\cdot y $ ($x$ durch Multiplikation von $y$ getrennt), dann kann mittels $\frac{y'}{y}$ und ein $\int_{x_0}^{x_1}dx$ die Funktion $y$ gefunden werden.

\subsubsection{Linearterm}
Ein Sonderfall für DGL 1. Ordnung: $y' = f(ax +bx+c)$ Substituieren $z := ax + bx +c$ und mit $z' = a + bf(z)$ ergänzen. Anschliessend: \[
\int_{x_0}^{x}\frac{z'}{a+bf(z)}d\tilde{x} = \int_{x_0}^{x}1d\tilde{x} \Rightarrow \int_{z_0}^{z}\frac{1}{a +bf(\tilde{z})}d\tilde{z} = \int_{x_0}^{x}1d\tilde{x}
\] 


\subsubsection{Gleichgradigkeit}
Ein Sonderfall für DGL 1. Ordnung: $y' = f(\frac{y}{x})$ Substituieren $z := \frac{y}{x}$ und mit $z' = \frac{1}{x}(y(z) - z)\qquad z_0 = \frac{y_0}{x_0}$ ergänzt.\\
\textbf{Beispiel} 
\begin{align*}
	y' &= 2 \cdot \frac{y}{x} + 1 \xRightarrow{z := \frac{y}{x}} y' = 2z + 1 & 	z' &= \frac{2z + 1 - z}{x}  \\
	\int \frac{1}{z+1}dz &= \int \frac{1}{x}dx & | \int \dots \\
	z &= Kx - 1 \xRightarrow{\frac{y}{x} := z} y = Kx^2 - x
\end{align*}

\subsubsection{Linear DGL n. Ordnung}
Um lineare DGL mit Konstanten Koeffizienten zu berechnen, kann das charakteristische Polynom verwendet werden:
\[
\cdots + y'' + a_1y' + a_0y = f(x) \quad\rightarrow\quad \cdots + \lambda^2 + a_1\lambda + a_0 = 0
\]
aus $\lambda_i$ können die $\mathbb{H}$-Lösungen berechnet werden. Dabei gilt pro $\lambda$:
\begin{enumerate}[nosep]
	\item $\lambda \in \mathbb{R}$, Vielfachheit $r$ = Anzahl gleicher $\lambda$:\\
	$\mathbb{H}_i = x^{r-1} \cdot e^{\lambda_i x}$
	\item $\lambda \in \mathbb{C}\backslash\mathbb{R}$ , Vielfachheit $k$ = Anzahl gleicher $\lambda$ (Nur \textbf{positiver} Imaginärteil verwenden: $\lambda = \alpha + j\beta$):\\
	$\mathbb{H}_i = x^{k-1} \cdot e^{\alpha x}\cdot \cos(\beta x)$\\
	$\mathbb{H}_{i+1} = x^{k-1} \cdot e^{\alpha x}\cdot \sin(\beta x)$  \\
\end{enumerate}

\noindent Die $\mathbb{H}$-Lösungen ist nun die lineare Kombination von allen $\mathbb{H}_i$.\\
\textbf{Beispiel: Ordnung $n=3$}
\begin{align*}
	1\lambda^4 + 2\lambda^2 + 1 = 0 &\qquad\Rightarrow \mathbb{L} = \left\{+j;+j;\xcancel{-j};\xcancel{-j}\right\} (\smalloverbrace{0}^\alpha + j\smalloverbrace{1}^\beta) \\
	\xRightarrow[]{} \lambda \in \mathbb{C}; k=2 &\qquad \mathbb{H}_1 = 1 e^{0x}\cos(1x); \quad \mathbb{H}_2 = 1xe^{0x}\cos(1x) \\
		                                         &\qquad \mathbb{H}_3 = 1 e^{0x}\sin(1x); \quad \mathbb{H}_4 = 1xe^{0x}\sin(1x) \\
	\mathbb{L}_\mathbb{H} &= \left\{A\mathbb{H}_1 + B\mathbb{H}_2+ C\mathbb{H}_3+ D\mathbb{H}_4 \right\}_{A,B,C,D\in\mathbb{R}}
\end{align*}

\noindent Anschliessend können die $\mathbb{P}$-Lösungen mittels \textbf{Faltungsintegral} berechnet werden (wenn möglich $x_0 = 0$) verwenden:
\[
y_p(x) = \int_{x_0}^{x}g(x + x_0 - t)\cdot\underbrace{f(t)}_\text{Störterm}dt
\]
\noindent Die Funktion $g(x_0)\in\mathbb{H}$ mit GLS $\left(\begin{smallmatrix}	g(x_0)\\ g^{(n-1)}(x_0)\\ g^{(n)}(x_0) \end{smallmatrix}\right) \eqi \underbrace{\left(\begin{smallmatrix}	0\\ 0\\ 1 \end{smallmatrix}\right)}_\text{{Binär } 1}$ oder entsprechenden Startvektor bestimmen. Konkret wird die $\mathbb{H}$-Lösung verwendet und mittels Anfangswerte, $A,B,...$ berechnet und somit eine konkrete Funktion $g$ gefunden.\\

\noindent Alternativ kann auch eine Tabelle verwendet werden. Den Störterm Analysieren und in einer der Form bringen. Dafür wird eine \underline{Testzahl} $\alpha$ und/oder $\beta$ generiert um den Störterm mit einer der folgenden Ausdrücke zu bekommen: Polynom $p$ von $m$ Grades, \textbf{Resonanz} ist $r$-fache Lösung der char. Gleichung:\\
\begin{tabular}{p{4.5cm} | p{4.5cm}}
	\multicolumn{2}{c}{\textbf{Störterm} = $e^{\alpha x}p_m(x)$} \\[0.2cm]
	\toprule
	$\alpha \notin \lambda$: & $\alpha\in\lambda$:  \\
	$\mathbb{P} = e^{\alpha x} q_m(x)$ & $\mathbb{P} = e^{\alpha x}x^rq_m(x)$ \\
	\hline
	\multicolumn{2}{c}{\textbf{Störterm} = $e^{\alpha x}\left[p_m(x)\cos(\beta x) + q_m(x)\sin(\beta x)\right]$} \\[0.2cm]
	\toprule
	$\alpha + j\beta \notin \lambda$: & $\alpha + j\beta \in \lambda$:  \\
	$\mathbb{P} = e^{\alpha x}[r_m(x)\cos(\beta x) +$ $ s_m(x)\sin(\beta x)]$ & $\mathbb{P} = e^{\alpha x}x^r [r_m(x)$  $\cos(\beta x) + s_m(x)\sin(\beta x)]$ \\
\end{tabular}\\

\noindent Die DGL Lösung ist \underline{immer} eine Summe von $\mathbb{L} = \mathbb{H} + \mathbb{P}$

\subsection{Orthogonaltrajektorien}
Senkrechte Kurven, welche Originalkurven immer senkrecht schneiden.
\begin{enumerate}[nosep]
	\item Ableiten und Parameter eliminieren, DGL als r.h.s notieren 
	\item Orthogonale Kurven $\tilde{y}$ mit Negative Reziprok definieren.
\end{enumerate}

\noindent\textbf{Beispiel}:
\begin{align*}
	\text{1. } y &= cx & | \cdot \frac{d}{dx}\quad | c=\cdots\\
	y' &= c \quad c = \frac{y}{x} \Rightarrow y' = \frac{y}{x}\\
	\text{2. } \tilde{y}' &= -\frac{x}{y}
\end{align*}

